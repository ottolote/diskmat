\documentclass[a4paper]{article}
\usepackage[english]{babel}
\usepackage[utf8]{inputenc}

% mathermatics
\usepackage{amssymb} % useful math symbols
\usepackage{amsmath} % more useful math
\usepackage{bm} % for bold matrices

% references and equation numbering
\usepackage{hyperref}
\usepackage{cleveref}
\usepackage{autonum}

% graphics
\usepackage{graphicx}
\usepackage{float}    % for more accurate graphics placement
\usepackage{fancyhdr} % for top header

% formatting
\usepackage{enumitem} % provides easy change of labels in enumerate environment
\usepackage[top=3cm, bottom=3cm, width=17cm]{geometry} % for smaller page margins

% colors
\usepackage{xcolor}

%% coding
% 
% see https://github.com/gpoore/minted for info
% uncomment this after you have completed the required installation
%
%\usepackage{minted} 
%\definecolor{codeBgColor}{RGB}{240,240,240}
%

\newcommand{\matr}[1]{\bm{#1}}


% very simple alias, \ex{} becomes the same as \subsubsection*{}
% TIP: remove the * in the line below if you want it numbered
\newcommand{\ex}[1]{\subsubsection*{#1}}




%Begining of the document
\begin{document}

\pagestyle{fancy} % use pagestyle with simple header (from fancyhdr)

%\pagenumbering{gobble} % uncomment to remove pagenumbering (in case of single page document)
\fancyhead[L]{TMA4140 Diskret Matematikk}
\fancyhead[C]{\textbf{Øving 7}}
\fancyhead[R]{Otto Lote 748704}


\subsection*{Section 5.1}
\ex{Exercise 4}

\begin{enumerate}[label=\alph*)]
    \item \(P(1): 1^3 = (\frac{1(1+1)}{2})^2\)

    \item Basis step:

        \(P(1): 1^3 = (\frac{1(1+1)}{2})^2 = 1\) \\
        \(P(1)\) is true, which completes the basis step of a proof by
            induction for \(P(k)\)

    \item The inductive hypothesis consists of two parts: \\
        - \(P(b)\) holds true \\
        - \(P(k) \rightarrow P(k+1)\) holds true\\
        Then \(P(k), \quad \forall k > b\)

        In other words, if \(P\) is true for the first step, and \(P\) holds
            true for an arbitrary step implies \(P\) holds true for the next
            step, then \(P\) holds true for all steps. \\

    \item You need to prove the first step \(P(b)\), and then you need to prove
        \(P(k) \rightarrow P(k+1)\)

    \item {
            \begin{align}
                \intertext{Basis step:}
                1^3 = (1(1+1)/2)^2 = 1 \\
                \intertext{LHS = RHS}
                \intertext{Inductive step:}
                \sum_{n=1}^k{n^3} &= \Big(\frac{k(k+1)}{2}\Big)^2 \label{eq:4e_k}\\
                \intertext{We assume \(P(k)\) is true for an arbitrary integer
                    \(k\). We can replace \(k\) with \(k+1\). Out goal is to
                    show that if \(P(k)\) holds then \(P(k+1)\) must hold}
                \sum_{n=1}^{k+1}{n^3} &= \Big(\frac{(k+1)((k+1)+1)}{2}\Big)^2 \\
                \sum_{n=1}^{k}{n^3} + (k+1)^3 &= \Big(\frac{(k+1)(k+2)}{2}\Big)^2 \\
                &= \Big(\frac{(k^2+3k+2)}{2}\Big)^2 \\
                &= \Big(\frac{(k(k+1))}{2} + (k+1)\Big)^2 \\
                \sum_{n=1}^{k}{n^3} + (k+1)^3 &= \Big(\frac{(k(k+1))}{2}\Big)^2
                    + k(k+1)(k+1) + (k+1)^2 \\
                \intertext{We subtract (\ref{eq:4e_k}) from the equation and have}
                (k+1)^3 &= k(k+1)^2 + (k+1)^2 \\
                (k+1)^3 &= (k+1)(k+1)^2 = (k+1)^3 \\
                \intertext{LHS = RHS. This means that if \(P(k)\) holds true,
                    then \(P(k+1)\) must also be true, which by the inductive
                    hypothesis means that \(\forall k \geq 1, \quad P(k)\) holds true}
            \end{align}
    }
\end{enumerate}


\ex{Exercise 6}

\begin{align}
    \intertext{Basis step:}
    1\cdot 1! &= (1 + 1)! - 1 \\
    1 &= (2)! - 1 = 2 - 1 = 1\\
    \intertext{LHS = RHS, so \(P(1)\) holds true}
    \intertext{Inductive step:}
    \sum_{n = 1}^k{n\cdot n!} &= (k + 1)! - 1 \label{eq:1-6-k}\\
    \intertext{We assume that \(P(k)\) holds for all \(k > 1\), and replace
        \(k\) with \(k+1\)}
    \sum_{n = 1}^{k+1}{n\cdot n!} &= ((k+1) + 1)! - 1 \\
    (k+1)\cdot(k+1)! + \sum_{n = 1}^{k}{n\cdot n!}  &= (k+2)! - 1 \\
    (k+1)\cdot(k+1)! + \sum_{n = 1}^{k}{n\cdot n!}  &= (k+2)(k+1)! - 1 \\
    \intertext{We subtract (\ref{eq:1-6-k}) from the equation and get}
    (k+1)\cdot(k+1)! +  &= (k+2)(k+1)! - 1 - \big( (k+1)! - 1\big)\\
    (k+1)\cdot(k+1)! +  &= (k+2)(k+1)!  -  (k+1)! \\
    (k+1)\cdot(k+1)! +  &= k(k+1)! + 2(k+1)! - (k+1)! \\
    (k+1)\cdot(k+1)! +  &= k(k+1)! + (k+1)!  \\
    (k+1)\cdot(k+1)! +  &= (k+1)\cdot(k+1)!  \\
    \intertext{LHS = RHS, so \(P(k)\) must hold for all \(k > 1\)}
\end{align}


\newpage
\ex{Exercise 14} 

\begin{enumerate}[label=\alph*)] 
    \item 
        \begin{align}
            \intertext{\(P(k)\):}
            \sum_{n=1}^k{ n2^n} &= (k-1)2^{k+1} + 2 \label{eq:1-14-k}\\
            \intertext{Basis step:}
            1\cdot 2^1 = (1-1)2^2{1+1} + 2 = 2\\
            \intertext{LHS = RHS, so \(P(1)\) holds true}
            \intertext{Induction step:}
            \intertext{Replacing \(n\) with \(k+1\) in (\ref{eq:1-14-k}) gives us}
            \sum_{n=1}^{k+1}{ n2^n} &= ((k+1)-1)2^{(k+1)+1} + 2 \\
            (k+1)2^{k+1} + \sum_{n=1}^{k}{ n2^n} &= k2^{k+2} + 2 \\
            (k+1)2^{k+1} + \sum_{n=1}^{k}{ n2^n} &= 2k2^{k+1} + 2 \\
            \intertext{We subtract (\ref{eq:1-14-k}) from the equation and get}
            (k+1)2^{k+1} &= 2k2^{k+1} + 2 - \big((k-1)2^{k+1} + 2\big) \\
            (k+1)2^{k+1} &= 2k2^{k+1} - (k-1)2^{k+1} \\
            (k+1)2^{k+1} &= (2k - (k-1)) 2^{k+1}\\
            (k+1)2^{k+1} &= (k+1)2^{k+1}\\
            \intertext{LHS = RHS, so \(P(k)\) must hold for all \(k > 1\)}
        \end{align}
\end{enumerate}


%\vspace{1em}
%\subsection*{Section 5.2}
%
%\ex{Exercise 4}
%
%
%\ex{Exercise 14} 



\subsection*{Section 5.3}
\ex{Exercise 12}

\begin{align}
    \intertext{Basis step:}
    f_1^2 &= f_1f_2 = 1 \\
    \intertext{Inductive step:}
    \sum_{n=1}^{k+1}{f_k^2} &= f_kf_{k+1} + f_{k+1}^2 = f_{k+1}(f_k + f_{k+1}) \\
    &= f_{n+1}f_{n+2} \\
\end{align}

\ex{Exercise 18}

\begin{align}
    \intertext{Basis step:}
    \matr{A}^1 &= 
        \begin{bmatrix}
            f_2 & f_1 \\
            f_1 & f_0 \\
        \end{bmatrix} =
        \begin{bmatrix}
            1 & 1 \\
            1 & 0 \\
        \end{bmatrix} = \matr{A} \\
    \intertext{Inductive step:}
    \matr{A}^{k+1} &= \matr{A}\matr{A}^k = 
        \begin{bmatrix}
            1 & 1 \\
            1 & 0 \\
        \end{bmatrix}
        \begin{bmatrix}
            f_{k+1} & f_k \\
            f_k & f_{k-1} \\
        \end{bmatrix} = 
        \begin{bmatrix}
            f_{k+1} + f_k & f_k + f_{k-1} \\
            f_{k+1} & f_k \\
        \end{bmatrix} = 
        \begin{bmatrix}
            f_{k+2} & f_{k+1} \\
            f_{k+1} & f_k \\
        \end{bmatrix}
\end{align}



\subsection*{Section 5.4}
\ex{Exercise 3}

The calls can be unrolled in the following manner

\begin{verbatim}
gcd(8,13)
8 != 0 -> gcd(13 mod 8, 8)
gcd(5,8)
5 != 0 -> gcd(8 mod 5, 5)
gcd(3,5)
3 != 0 -> gcd(5 mod 3, 3)
gcd(2,3)
2 != 0 -> gcd(3 mod 2, 2)
gcd(1,2)
1 != 0 -> gcd(2 mod 1, 1)
gcd(1,1)
1 != 0 -> gcd(1 mod 1, 1)
gcd(0,1)
1 = 0 -> return 1
-----------------------
gcd(8,13) = 1
\end{verbatim}

%% uncomment this if you need references. Edit the .bbl file with your references
%% and use "\cite{bibitem-label}" to cite
%\bibliography{template}

\end{document}

