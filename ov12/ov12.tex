\documentclass[a4paper]{article}
\usepackage[english]{babel}
\usepackage[utf8]{inputenc}

% mathermatics
\usepackage{amssymb} % useful math symbols
\usepackage{amsmath} % more useful math
\usepackage{bm} % for bold matrices

% references and equation numbering
\usepackage{hyperref}
\usepackage{cleveref}
\usepackage{autonum}

% graphics
\usepackage{graphicx}
\usepackage{float}    % for more accurate graphics placement
\usepackage{fancyhdr} % for top header

% formatting
\usepackage{enumitem} % provides easy change of labels in enumerate environment
\usepackage[top=3cm, bottom=3cm, width=17cm]{geometry} % for smaller page margins

% colors
\usepackage{xcolor}

%% coding
% 
% see https://github.com/gpoore/minted for info
% uncomment this after you have completed the required installation
%
%\usepackage{minted} 
%\definecolor{codeBgColor}{RGB}{240,240,240}
%

\newcommand{\matr}[1]{\bm{#1}}


% very simple alias, \ex{} becomes the same as \subsubsection*{}
% TIP: remove the * in the line below if you want it numbered
\newcommand{\ex}[1]{\subsubsection*{#1}}




%Begining of the document
\begin{document}

\pagestyle{fancy} % use pagestyle with simple header (from fancyhdr)

%\pagenumbering{gobble} % uncomment to remove pagenumbering (in case of single page document)
\fancyhead[L]{TMA4140 Diskret Matematikk}
\fancyhead[C]{\textbf{Øving 12}}
\fancyhead[R]{Otto Lote 748704}


\subsection*{Section 13.1}
\ex{Exercise 4}

\begin{enumerate}[label=\alph*)]
    \item Using a bottom down approach we start with \(11100A \to 111000\).
        Using \(S \to 00A\) we get \(111S \to 11100A \to 111000\). If we use
        \(S \to 1S\) three times we get \(S \to 1S \to 11S \to 111S \to 11100A
        \to 111000\), which shows that 111000 is a production of \(G\).
    \item \(S\) is the starting symbol, which means all our productions belong
        to \(S \to 1S\) or \(S \to 00A\). Since \(A \to 0A\) or \(A \to 0\) are
        the only possibilities here there is no way to have a 1 to the right of
        a 0.
    \item Given an \(S\) we have that \(S \to 00A\). Since \(A \to 0A\) we can
        use this to insert a 0 in front of an A. Since \(S \to 1S\) we can
        always add a 1 to the left of a valid \(S\). Summarized this means the
        shortest \(S \to 00A \to 000\) can be expanded left or right with 1s
        and 0s respectively. In other words, the phrase belongs to \(G\) iff it
        only exists of the symbols 0 or 1, all 1s are left of all 0s, and there
        are at least three 0s. 
\end{enumerate}


\ex{Exercise 6} 

\begin{enumerate}[label=\alph*), start=4] % d, e
    \item From a top down approach we start with \(S \to B\). The first option
        is \(B \to b\) giving us \(S \to B \to b\). The other option is \(B \to
        bB\). The effect of \(B \to bB\) is an inserted \(b\) before a given
        \(B\). \(B \to bB\) is repeatable infinitely, giving us a language
        consisting of the shortest phrase that stems from \(S \to B\), which is
        \(b\), with \(n \in \mathbb{N}\) insertions of \(b\). (In other words,
        a phrase consisting of 1 or more \(b\)s and no other symbols).

        Starting with \(S \to AA\) we have the shortest phrase \(S \to AA \to
        aaA \to aaaa\). The result of \(A to aaA\) is an insertion of \(aa\)
        before a given \(A\). This is repeatable, meaning the language consists
        of \(aaaa\) with inserted \(aa\) \(k \in \mathbb{N}\) times. In other
        words, all phrases consisting of only the symbol \(a\) of length \(n
        \leq 4\) where \(n\) is even.
    \item Using a top down approach the only starting symbol is \(S \to AB\).
        We see that \(A \to aAb\) is repeatable, and the result is append \(b\)
        and insert \(a\). \(B \to bBa\) is also repeatable, but inserts \(b\)
        and appends \(a\). This means the language consists of separated
        groupings of \(a\)s and \(b\)s where the number of \(a\)s \(n \in
        \mathcal{N}\) is the same as the number of \(b\)s and the order of the
        groupings are arbitrary.
\end{enumerate}

\ex{Exercise 20}



\subsection*{Section 13.3}

\ex{Exercise 8}

\begin{enumerate}[label=\alph*)]
    \item 
\end{enumerate}

\ex{Exercise 10}

\begin{enumerate}[label=\alph*)]
    \item 
\end{enumerate}

\ex{Exercise 12}

\begin{enumerate}[label=\alph*)]
    \item 
\end{enumerate}

\ex{Exercise 16}

\begin{enumerate}[label=\alph*)]
    \item 
\end{enumerate}

\ex{Exercise 22}

\begin{enumerate}[label=\alph*)]
    \item 
\end{enumerate}

\ex{Exercise 24}

\begin{enumerate}[label=\alph*)]
    \item 
\end{enumerate}

\ex{Exercise 36}

\begin{enumerate}[label=\alph*)]
    \item 
\end{enumerate}


%% uncomment this if you need references. Edit the .bbl file with your references
%% and use "\cite{bibitem-label}" to cite
%\bibliography{template}

\end{document}

