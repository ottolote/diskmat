\documentclass[a4paper]{article}
\usepackage[english]{babel}
\usepackage[utf8]{inputenc}

% mathermatics
\usepackage{amssymb} % useful math symbols
\usepackage{amsmath} % more useful math

% references and equation numbering
\usepackage{hyperref}
\usepackage{cleveref}
\usepackage{autonum}

% graphics
\usepackage{graphicx}
\usepackage{float}    % for more accurate graphics placement
\usepackage{fancyhdr} % for top header

% formatting
\usepackage{enumitem} % provides easy change of labels in enumerate environment
\usepackage[top=3cm, bottom=4cm, width=17cm]{geometry} % for smaller page margins

% colors
\usepackage{xcolor}

%% coding
% 
% see https://github.com/gpoore/minted for info
% uncomment this after you have completed the required installation
%
\usepackage{minted} 
\definecolor{codeBgColor}{RGB}{240,240,240}
%



% very simple alias, \ex{} becomes the same as \subsubsection*{}
% TIP: remove the * in the line below if you want it numbered
\newcommand{\ex}[1]{\subsubsection*{#1}}




%Begining of the document
\begin{document}

\pagestyle{fancy} % use pagestyle with simple header (from fancyhdr)

%\pagenumbering{gobble} % uncomment to remove pagenumbering (in case of single page document)
\fancyhead[L]{TMA4140 Diskret Matematikk}
\fancyhead[C]{\textbf{Øving 3}}
\fancyhead[R]{Otto Lote 748704}


\subsection*{Section 3.1}
\ex{Exercise 53}

\begin{enumerate}[label=\alph*)]
    \item Total: 2 quarters, 1 penny (51c)

    \item Total: 2 quarters, 1 dimes, 1 nickel, 4 pennies (69c)

    \item Total: 3 quarters, 1 penny (76c)

    \item Total: 2 quarters, 1 dime (60c)
\end{enumerate}


\ex{Exercise 55}

\begin{enumerate}[label=\alph*)]
    \item Total: 2 quarters, 1 penny (51c)

    \item Total: 2 quarters, 1 dime, 9 pennies (69c)

    \item Total: 3 quarters, 1 penny (76c)

    \item Total: 2 quarters, 1 dime (60c)

        The fewest coins are given when \((n \, \mathrm{mod}25) \, \mathrm{mod} 10 \geq 5\)

\end{enumerate}


\ex{Exercise 56}

Proof by counterexample:

15c using the greedy algorithm would draw one 12c coin and 3 pennies, which is four coins, when one dime and one nickel would suffice. 


\subsection{Section 3.2}
\ex{Exercise 27}

\begin{enumerate}[label=\alph*)]
    \item a

    \item b
\end{enumerate}


\ex{Exercise 30} % c e

\begin{enumerate}[label=\alph*)]
    \item

    \item
\end{enumerate}


\ex{Exercise 34}

\begin{enumerate}[label=\alph*)]
    \item

    \item
\end{enumerate}


\ex{Exercise 42}

\begin{enumerate}[label=\alph*)]
    \item

    \item
\end{enumerate}


\subsection*{Section 4.1}
\ex{Exercise 11}

\begin{enumerate}[label=\alph*)]
    \item

    \item
\end{enumerate}



%% uncomment this if you need references. Edit the .bbl file with your references
%% and use "\cite{bibitem-label}" to cite
%\bibliography{template}

\end{document}

