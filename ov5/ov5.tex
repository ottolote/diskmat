\documentclass[a4paper]{article}
\usepackage[english]{babel}
\usepackage[utf8]{inputenc}

% mathermatics
\usepackage{amssymb} % useful math symbols
\usepackage{amsmath} % more useful math

% references and equation numbering
\usepackage{hyperref}
\usepackage{cleveref}
\usepackage{autonum}

% graphics
\usepackage{graphicx}
\usepackage{float}    % for more accurate graphics placement
\usepackage{fancyhdr} % for top header

% formatting
\usepackage{enumitem} % provides easy change of labels in enumerate environment
\usepackage[top=3cm, bottom=4cm, width=17cm]{geometry} % for smaller page margins

% colors
\usepackage{xcolor}

%% coding
% 
% see https://github.com/gpoore/minted for info
% uncomment this after you have completed the required installation
%
%\usepackage{minted} 
%\definecolor{codeBgColor}{RGB}{240,240,240}
%



% very simple alias, \ex{} becomes the same as \subsubsection*{}
% TIP: remove the * in the line below if you want it numbered
\newcommand{\ex}[1]{\subsubsection*{#1}}




%Begining of the document
\begin{document}

\pagestyle{fancy} % use pagestyle with simple header (from fancyhdr)

%\pagenumbering{gobble} % uncomment to remove pagenumbering (in case of single page document)
\fancyhead[L]{TMA4140 Diskret Matematikk}
\fancyhead[C]{\textbf{Øving 4}}
\fancyhead[R]{Otto Lote 748704}


\subsection*{Section 4.4}
\ex{Exercise 21}

\begin{align}
    m &= 2 \times 3 \times 5 \times 11 = 330 \\
    M_1 &= \frac{330}{2} = 165 \equiv 1 \pmod 2 \\
    M_2 &= \frac{330}{3} = 110 \equiv 2 \pmod 3 \\
    M_3 &= \frac{330}{5} = 66 \equiv 1 \pmod 5 \\
    M_4 &= \frac{330}{11} = 30 \equiv 8\pmod{11}\\
    \intertext{\(a_1M_1y_1 + a_2M_2y_2 + ...\) is a simultaneous solution where
        \(y_n\) is an inverse of \(M_k\) such that \(M_ky_k \equiv 1 \pmod {m_k}\)}
    \intertext{This yields the following \(y_k\)}
    y_1 &= 1, y_2 = 2, y_3 = 1, y_4 = 7 \\
    \intertext{Which gives us the solution}
    1\times 165 \times 1 &+ 2\times 110 \times 2 + 3 \times 66 \times 1
        + 4 \times 30 \times 7 = 165 + 440 + 198 + 840 = 1643 \\
    \intertext{We know the solution is \(\pmod{m}\)}
    1643 \equiv 323 \pmod{330} \\
    \intertext{which gives us all the solutions}
    x = 323 + 330n, \quad \forall n \in \mathbb{N}
\end{align}


\ex{Exercise 33}

\begin{align}
    7^{121} \mod 13 &= 7^{12 \times 10 + 1} \mod 13  \\
    &= 7^{12}\times 7^{10} \times 7 \mod 13 \\
    &= 1 \pmod{13} \times 1 \pmod{13} \times 7 \mod 13 = 7\\
\end{align}


\ex{Exercise 37} % a

\begin{enumerate}[label=\alph*)] 
    \item 
        \begin{align}
            2^{340} \mod 11 &= (2^{10})^{34} \mod 11 \\
            \intertext{We know that \(2^{10}\ \pmod{11} \equiv 1 \pmod {11}\)} 
            &\equiv 1^{34} \pmod {11} = 1 \pmod {11}\\
        \end{align}
\end{enumerate}


\vspace{1em}
\subsection*{Section 6.1}

\ex{Exercise 27}
We have three possibilities per state and 50 states, so the number of possibilities is \(3^{50}\)


\ex{Exercise 44} 
Choosing 4 from 10 can be arranged in \(\binom{4}{10} = 5040\) ways. Out of
these ways of seating the group every sequence (1-2-3-4) has three other
equivalent permutations (2-3-4-1, 3-4-1-2, 4-1-2-3), so the total number of
ways to seat 10 people around a circular table holding 4 people is \(5040/4 =
1260\)



\subsection*{Section 6.2}
\ex{Exercise 10}
The midpoint of a line between two points \((a_x,a_y)\) and \((b_x,b_y)\) is \((\frac{a_x+b_x}{2},\frac{a_y+b_y}{2})\) which is an integer point when \(a_x + b_x\) both and \(a_y + b_y\) are even, which is when (\(a_x\) and \(b_x\)) and (\(a_y\) and \(b_y\)) are both of the same parity. Since we have 5 points at least two of them has four coordinates that are of the same parity (because of the pidgeonhole principle).

\ex{Exercise 18}

\begin{enumerate}[label=\alph*)]
    \item 
        \begin{align}
            n_{male} &= n - n_{female} \\
            n_{male} &\leq 4 \implies n_{female} \geq 5 \\
            \intertext{and vice versa}
        \end{align}

    \item
        \begin{align}
            n_{male} &= n - n_{female} \\
            n_{male} &\geq 3 \implies n_{female} \leq 6 \\
            n_{female} &\geq 7 \implies n_{male} \leq 2 \\
            \intertext{and vice versa}
        \end{align}
\end{enumerate}



\subsection*{Section 6.3}
\ex{Exercise 13}
Since men and women alternate the line can be viewed as two lines, one for women and one for men, that are interleaved. Arranging the whole line will be the same as arranging each line and then interleaving them. The number of possibilities is \(n!\) per line, but for every possibility in one line we have \(n!\) in the other, so the total amount of possibilities is \(n!\times n! = (n!)^2\)

\ex{Exercise 19} % b, c

\begin{enumerate}[start = 2, label=\alph*)]
    \item The number of possibilities that a flip sequence comes up heads exactly 2 times is like choosing 2 from a set of 10. The number of possibilities is \(\binom{2}{10} = 90\)

    \item A flip sequence that contain at most three tail is the sum of choosing 3 from 10, choosing 2 from 10 and choosing 1 from 10. The total number of possibilities is \(\binom{3}{10} + \binom{2}{10} + \binom{1}{10} = 120 + 45 + 10 = 175\)
\end{enumerate}

\ex{Exercise 34}
The committee can have 0, 1 or 2 men. 
This means the number of possibilities is \(\binom{15}{6} + \binom{15}{5}\binom{10}{1} + \binom{15}{4}\binom{10}{2} = 5005 + 3003\times 10 + 1365\times 45 = 96460\)


\subsection*{Section 6.4}
\ex{Exercise 9}
\begin{align}
    \intertext{We have the binomial theorem}
    (x+y)^n = \sum_{j=0}^n{\binom{n}{j}x^{n-j}y^j} \\
    \intertext{Inserting \((2x-3y)^{200}\) gives us}
    (2x-3y)^{200} = \sum_{i=0}^{200}\binom{200}{i}(2x)^i(-3y)^{200-i} \\
    \intertext{For \(x^{101}y^{99}\) we get}
    \binom{200}{99}(2x)^{101}(-3y)^{99} = \underbrace{-2^{101}3^{99}\binom{200}{99}}_{\text{the coefficient in question}}x^{101}y^{99} \\
\end{align}

%% uncomment this if you need references. Edit the .bbl file with your references
%% and use "\cite{bibitem-label}" to cite
%\bibliography{template}

\end{document}

