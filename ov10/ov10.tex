\documentclass[a4paper]{article}
\usepackage[english]{babel}
\usepackage[utf8]{inputenc}

% mathermatics
\usepackage{amssymb} % useful math symbols
\usepackage{amsmath} % more useful math
\usepackage{bm} % for bold matrices

% references and equation numbering
\usepackage{hyperref}
\usepackage{cleveref}
\usepackage{autonum}

% graphics
\usepackage{graphicx}
\usepackage{float}    % for more accurate graphics placement
\usepackage{fancyhdr} % for top header

% formatting
\usepackage{enumitem} % provides easy change of labels in enumerate environment
\usepackage[top=3cm, bottom=3cm, width=17cm]{geometry} % for smaller page margins

% colors
\usepackage{xcolor}

%% coding
% 
% see https://github.com/gpoore/minted for info
% uncomment this after you have completed the required installation
%
%\usepackage{minted} 
%\definecolor{codeBgColor}{RGB}{240,240,240}
%

\newcommand{\matr}[1]{\bm{#1}}


% very simple alias, \ex{} becomes the same as \subsubsection*{}
% TIP: remove the * in the line below if you want it numbered
\newcommand{\ex}[1]{\subsubsection*{#1}}




%Begining of the document
\begin{document}

\pagestyle{fancy} % use pagestyle with simple header (from fancyhdr)

%\pagenumbering{gobble} % uncomment to remove pagenumbering (in case of single page document)
\fancyhead[L]{TMA4140 Diskret Matematikk}
\fancyhead[C]{\textbf{Øving 10}}
\fancyhead[R]{Otto Lote 748704}


\subsection*{Section 10.3}
\ex{Exercise 37}

The graphs are isomorphic since both graphs are cyclic and have the same number
    of vertices all of degree 2. This means we can follow either of the graphs
    in its cycle and label their vertices as \(u_1 = v_1,\, u_2 = v_3,\, u_3 = v_5,\,
    ...\) and so on.

\ex{Exercise 42}

These graphs are not isomorphic since there is an edge between \(u_2\) and
    \(u_3\), which are vertices of degree 4, while the other two vertices of
    degree four \(v_2\) and \(v_4\) does not have an edge between them. 

\ex{Exercise 54} 

\begin{enumerate}[label=\alph*), start=3] % c
    \item If we build on the set of non-isomorphic simple graphs with 3
        vertices (there are two) adding a vertex can increase the degree of
        one, two or three existing vertices by one. The first graph in the set
        is a triangle. Adding a vertex to this graph can be done in three ways
        since the triangle is symmetric (new vertex has one edge, two edges, or
        three edges). 
        
        For the other graph in the set with 3 vertices (the line)
        There are two ways of adding a vertex with one edge, two ways of adding
        a vertex with two edges, and one way of adding a vertex with three
        edges. These are all non-isomorphic with our existing four vertex
        graphs, except when the added vertex has two edges that forms a
        triangle. This is a total of 3 + 4 = 7 non-isomorphic graphs with four
        vertices. 

\end{enumerate}

\subsection*{Section 10.4}

\ex{Exercise 26}

\begin{enumerate}[label=\alph*), start=2] % b
    \item Between c and d of length 3:
        From \(c\) we have an edge to \(b\), \(f\) and \(d\) that might be part
        of a path to \(d\) with length 3. 
        
        \(b\) has 3 paths with length 2 that reach \(d\); \((b, e, d)\), \((b, c,
            d)\)  and \((b, a, d)\).

        \(f\) has two paths with length 2 that reach \(d\); \((f, c, d)\) and
            \((f, e, d)\).

        \(d\) has 3 paths with length 2 that reach \(d\); \((d, e, d)\), \((d, c,
            d)\)  and \((d, a, d)\).

        This is a total of 8 paths, which corresponds to \(\bm{A^3}_{(3,4)}\)
            where \(\bm{A}\) is the adjacency matrix of the graph in figure 1.
\end{enumerate}

\ex{Exercise 30}

The sum of all degrees of the vertices in a simple undirected graph is always
    two times the number of edges (two vertices per edge). This means there are
    an even number of vertices of an odd degree. Since a simple undirected
    graph has a simple path between any two vertices it follows that a vertex
    of an odd degree must have a simple path to another vertex of an odd degree

\ex{Exercise 56}

If \(\bm{A}\) is an adjacency matrix of a graph and \(v\) and \(w\) are
    vertices of that graph, then the number of paths of length \(r\) is given
    by \(\bm{A}^r_{(v, w)}\). Calculating this expression, and increasing \(r\)
    by one if the result is 0 will eventually produce non-zero result where
    \(r\) is the shortest lenght of a path from \(v\) to \(w\).

\subsection*{Section 10.5}

\ex{Exercise 3}

The graphs does not have an Eurler Circuit since \(a\) and \(d\) are vertices
    of an odd degree. It does however have an Euler Path since the rest of the
    vertices are of an even degree.

An example is \(a, c, e, b, e, c, d, b, a, e, d\)

\ex{Exercise 30}

The graph does not have a hamilton cycle since there is only one edge from the
    right side to the left side of the graph.

\ex{Exercise 36}

The graph has several hamilton cycles. An example is \(a, d, g, h, i, e, f, c, b, (a)\).

\ex{Exercise 48}

Two fully connected subgraphs that share exactly one vertex is an example of a
    graph that does not have a Hamilton cycle since there is no way "back" from
    either subgraph after "crossing" the shared vertex. If both subgraphs have
    equal amount of vertices (and are fully connected) their vertices are of
    degree \((n-1)/2\) except the shared vertex of degree \(n-1\). A concrete
    example is two triangles that share only one vertex.

\subsection*{Section 10.6}

\ex{Exercise 3}

Since all edges are weighted positively we can use Dijkstras algorithm. 

\begin{tabular}{l | c c c c c c c c}
    node visited in step & a & b & c & d & e & f & g & z \\
    \hline 
    a & 0 & 4 & 3 \\
    c & 0 & 4 & 3 & 6 & 9 \\
    b & 0 & 4 & 3 & 6 & 9 \\
    d & 0 & 4 & 3 & 6 & 7 & 11 \\
    e & 0 & 4 & 3 & 6 & 7 & 11 & 12 \\
    f & 0 & 4 & 3 & 6 & 7 & 11 & 12 & 18 \\
    g & 0 & 4 & 3 & 6 & 7 & 11 & 12 & 16 \\
    z & 0 & 4 & 3 & 6 & 7 & 11 & 12 & 16 \\
\end{tabular}

\ex{Exercise 14}

Finding the shortest path by using a weighted graph can be done by uniformly
    weighting the graph.

\ex{Exercise 18}

No, the sum of two unique weights in a shortest path can still be the same as
    the length of another path.


%% uncomment this if you need references. Edit the .bbl file with your references
%% and use "\cite{bibitem-label}" to cite
%\bibliography{template}

\end{document}

