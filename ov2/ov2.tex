\documentclass[a4paper]{article}
\usepackage[english]{babel}
\usepackage[utf8]{inputenc}

% mathermatics
\usepackage{amssymb} % useful math symbols
\usepackage{amsmath} % more useful math

% graphics
\usepackage{graphicx}
\usepackage{float}    % for more accurate graphics placement
\usepackage{fancyhdr} % for top header

% formatting
\usepackage{enumitem} % provides easy change of labels in enumerate environment
\usepackage[top=3cm, bottom=4cm, width=17cm]{geometry} % for smaller page margins

% colors
\usepackage{xcolor}

%% coding
% 
% uncomment this after you have completed the required installation
% see https://github.com/gpoore/minted for info
%
%\usepackage{minted} 
%\definecolor{codeBgColor}{RGB}{240,240,240}
%



% very simple alias, \ex{} becomes the same as \subsubsection*{}
% TIP: remove the * in the line below if you want it numbered
\newcommand{\ex}[1]{\subsubsection*{#1}}




%Begining of the document
\begin{document}

\pagestyle{fancy} % use pagestyle with simple header (from fancyhdr)

%\pagenumbering{gobble} % uncomment to remove pagenumbering (in case of single page document)
\fancyhead[L]{TMA4140 Diskret Matematikk}
\fancyhead[C]{\textbf{Øving 2}}
\fancyhead[R]{Otto Lote 748704}


\vspace{2em}
\subsection*{Section 2.1}
\vspace{1em}

\ex{Exercise 5}
\( \forall x \big( x \in A \leftrightarrow x \in B  \big) \leftrightarrow A = B \)   
\begin{enumerate}[label=\alph*)]
    \item \( A = B \) since A contains all elements of B and vice versa

    \item \( ((1)) \in A \land ((1)) \in B \) but \( (1) \notin A \rightarrow A \neq B\) 

    \item \( \emptyset \) is an empty set, and is not equal to a set containing an empty set.
\end{enumerate}


\ex{Exercise 24}
\begin{enumerate}[label=\alph*)]
    \item No. The power set of a set always include a set of an empty set: 
        \( P(\emptyset) = (\emptyset) \) 
        meaning a power set of a set is never empty. 
        \( \forall A \big( (\emptyset) \in P(A)  \big) \rightarrow 
        P(A) \neq \emptyset\)

    \item Yes. \( P((a)) = (\emptyset, (a)) \)

    \item No. \( P((\emptyset, a)) = 
        \big(\emptyset, (a), (\emptyset), (a, \emptyset)\big) \)
        The set in question is missing a \((\emptyset)\)

    \item Yes. \( P((a, b)) = \big(\emptyset, (a), (b), (a,b)\big) \) 
\end{enumerate}



\vspace{2em}
\subsection*{Section 2.2}
\vspace{1em}

\ex{Exercise 18}
\begin{enumerate}[label=\alph*), start=3]
    \item{ 
        \( (A-B) \equiv \forall x \mid x \in A \land x \notin B \)

        \( (A-B) -C \equiv \forall x \mid x \in (A-B) \land x \notin C \)

        \( (A-B) -C \equiv \forall x \mid x \in A \land x \notin B \land x \notin C \)

        \( (A-C) \equiv \forall x \mid x \in A \land x \notin C \)

        Since all elements of \( (A-B)-C \) are in \( (A-C) \), 
            \( (A-B)-C \subseteq (A-C) \)
    }

    \item{ 
        \( (A-C) \equiv \forall x \mid x \in A \land x \notin C \)

        \( (C-B) \equiv \forall x \mid x \in C \land x \notin B \)

        \( (A-C) \cap (C-B) \equiv \forall x \mid x \in (A-C) \land x \in (C-B) \)

        \( (A-C) \cap (C-B) \equiv \forall x \mid (x \in A \land x \notin C) \land 
            (x \in C \land x \notin B) \)

        Rewriting the paranthesis we get 

        \( (A-C) \cap (C-B) \equiv \forall x \mid x \in A \land 
            \underbrace{(x \notin C \land x \in C)}_{x \in \emptyset} \land x \notin B \)

        Clearly, \( (A-C) \cap (C-B) = \emptyset \)

    }
\end{enumerate}


\ex{Exercise 46}

\( \)



\vspace{2em}
\subsection*{Section 2.3}
\vspace{1em}

\ex{Exercise 12}
\begin{enumerate}[label=\alph*)]
    \item{
    }

    \item{
    }
\end{enumerate}


\ex{Exercise 38}
\begin{enumerate}[label=\alph*)]
    \item{
    }

    \item{
    }
\end{enumerate}


\ex{Exercise 42}
\begin{enumerate}[label=\alph*)]
    \item{
    }

    \item{
    }
\end{enumerate}


\vspace{2em}
\subsection*{Section 2.4}
\vspace{1em}

\ex{Exercise 12}
\begin{enumerate}[label=\alph*)]
    \item{ c
    }

    \item{
    }
\end{enumerate}


\ex{Exercise 33}
\begin{enumerate}[label=\alph*)]
    \item{ d
    }

    \item{
    }
\end{enumerate}


\vspace{2em}
\subsection*{Section 2.5}
\vspace{1em}

\ex{Exercise 16}
\begin{enumerate}[label=\alph*)]
    \item{
    }
\end{enumerate}


%% uncomment this if you need references. Edit the .bbl file with your references
%% and use "\cite{bibitem-label}" to cite
%\bibliography{template}

\end{document}

